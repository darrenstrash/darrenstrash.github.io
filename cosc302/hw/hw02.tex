\documentclass[letterpaper,11pt]{article}

\usepackage{fullpage}
\usepackage{amsmath}
\usepackage{amsthm}
\usepackage{hyperref}

\newtheorem{theorem}{Theorem}[section]

\begin{document}
\noindent{COSC 302: Analysis of Algorithms --- Spring 2018}

\noindent{Prof. Darren Strash}

\noindent{Colgate University} \\

\noindent{\bf Problem Set 2 --- Order of Growth and Asymptotic Analysis} \\

\noindent {\bf Due by 4:30pm Friday, Feb. 9, 2018 as a single pdf via Moodle (either generated via \LaTeX{}, or concatenated photos of your work). Late assignments are not accepted.} \\

This is an \emph{individual} assignment: collaboration (such as discussing problems and brainstorming ideas for solving them) on this assignment is highly encouraged, but the work you submit must be your own. Give information only as a tutor would: ask questions so that your classmate is able to figure out the answer for themselves. It is unacceptable to share any artifacts, such as code and/or write-ups for this assignment. If you work with someone in close collaboration, you must mention your collaborator on your assignment.

\emph{Suggested practice problems (not to be turned in): 3.1-2, 3.1-3, 3.1-5, 3.2-1, 3.2-4, 3-3 (a)}

\begin{enumerate}
\item Describe a $\Theta(n\log n)$-time algorithm that, given an array $A$ of $n$ integers and another integer $x$, determines whether or not there exist two elements in $A$ whose sum is exactly $x$. (\emph{Hint: do not use divide-and-conquer.})

Prove that your algorithm is correct by giving an invariant that holds throughout your algorithm, and arguing that it holds at initialization, throughout execution, and describe why this invariant implies correctness.

\item Problem 3.1-1 in CLRS.
\item Problem 3-2 (Big Oh, Big Omega, and Big Theta only) in CLRS.
\item Problem 3-4 (parts c,d,e,g) in CLRS.

\end{enumerate}

\end{document}
