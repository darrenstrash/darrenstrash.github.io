\documentclass[letterpaper,11pt]{article}

\usepackage{fullpage}
\usepackage{amsmath}
\usepackage{amsthm}
\usepackage{hyperref}

\newtheorem{theorem}{Theorem}[section]

\begin{document}
\noindent{COSC 302: Analysis of Algorithms --- Spring 2018}

\noindent{Prof. Darren Strash}

\noindent{Colgate University} \\

\noindent{\bf Problem Set 3 --- Solving Recurrences and Divide and Conquer I} \\

\noindent {\bf Due by 4:30pm Friday, Feb. 6, 2018 as a single pdf via Moodle (either generated via \LaTeX{}, or concatenated photos of your work). Late assignments are not accepted.} \\

This is an \emph{individual} assignment: collaboration (such as discussing problems and brainstorming ideas for solving them) on this assignment is highly encouraged, but the work you submit must be your own. Give information only as a tutor would: ask questions so that your classmate is able to figure out the answer for themselves. It is unacceptable to share any artifacts, such as code and/or write-ups for this assignment. If you work with someone in close collaboration, you must mention your collaborator on your assignment.

\emph{Suggested practice problems (not to be turned in): 4.3-1, 4.3-8, 4.4-2, 4.4-4, 4.4-6, 4-3}

\begin{enumerate}
\item Problem 4-1 from CLRS 3rd edition.
\item Using one of the methods discussed in lecture, give a tight asymptotic bound for the recurrence $T(n) = 8T(n/2) + n^3 \lg n$.
\item Problem 4.3-9 from CLRS 3rd edition.

\item \emph{Revisiting the pinePhone}. You are still hard at work testing the quality of pinePhones for Pineapple. 

\begin{enumerate}
\item Suppose now that you are given $3$ pinePhones. Present a strategy to find the highest safe rung with $\Theta(\sqrt[3]{n})$ pinePhone drops. 
\item Now show that you can find the highest safe rung with $4$ pinePhones with $\Theta(\sqrt[4]{n})$ pinePhone drops.
\item Suppose you continue your strategy to $k$ pinePhones: Define the number of pinePhone drops as a recurrence in terms of $n$, the number of rungs on the ladder, and $k$, the number of pinePhones you are given. \emph{Hint:} Make sure to give at least one base case. Do you need multiple base cases?
\end{enumerate}

{\bf Challenge:} Solve this recurrence without using substitution.


\end{enumerate}

\end{document}
