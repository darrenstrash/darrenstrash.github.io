\documentclass[letterpaper,11pt]{article}

\usepackage{fullpage}
\usepackage{amsmath}
\usepackage{amsthm}
\usepackage{hyperref}

\newtheorem{theorem}{Theorem}[section]

\begin{document}
\noindent{COSC 302: Analysis of Algorithms --- Spring 2018}

\noindent{Prof. Darren Strash}

\noindent{Colgate University} \\

\noindent{\bf Problem Set 1 --- Invariants and Induction} \\

\noindent {\bf Due by 4:30pm Friday, Feb. 2, 2018 as a single pdf via Moodle (either generated via \LaTeX{}, or concatenated photos of your work). Late assignments are not accepted.} \\

This is an \emph{individual} assignment: collaboration (such as discussing problems and brainstorming ideas for solving them) on this assignment is highly encouraged, but the work you submit must be your own. Give information only as a tutor would: ask questions so that your classmate is able to figure out the answer for themselves. It is unacceptable to share any artifacts, such as code and/or write-ups for this assignment. If you work with someone in close collaboration, you must mention your collaborator on your assignment.

\emph{Suggested practice problems (not to be turned in): 2.1-2, 2.1.4, 2.2-2, 2.2-4}

\begin{enumerate}
\item Problem 2.1-3 in CLRS, 3rd edition.
\item Problem 2-2 in CLRS, 3rd edition.
\item Prove by induction that for every non-negative integer $n$
\[\sum_{k=0}^{n} k^2 = \frac{n(n+1)(2n+1)}{6}.\]
\item Prove that given an unlimited supply of 6-cent coins, 10-cent coins, and 15-cent coins, one can make any amount of change larger than 29 cents.\footnote{This is problem 1 from Jeff Erickson's lecture notes on induction: \url{http://jeffe.cs.illinois.edu/teaching/algorithms/notes/98-induction.pdf}.}
\end{enumerate}

\end{document}
