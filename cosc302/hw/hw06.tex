\documentclass[letterpaper,11pt]{article}

\usepackage{fullpage}
\usepackage{amsmath}
\usepackage{amsthm}
\usepackage{hyperref}

\newtheorem{theorem}{Theorem}[section]

\begin{document}
\noindent{COSC 302: Analysis of Algorithms --- Spring 2018}

\noindent{Prof. Darren Strash}

\noindent{Colgate University} \\

\noindent{\bf Problem Set 6 --- Graph Traversals} \\
\noindent {\bf Due by 4:30pm Friday, March 30, 2018 as a single pdf via Moodle (either generated via \LaTeX{}, or concatenated photos of your work). Late assignments are not accepted.} \\

This is an \emph{individual} assignment: collaboration (such as discussing problems and brainstorming ideas for solving them) on this assignment is highly encouraged, but the work you submit must be your own. Give information only as a tutor would: ask questions so that your classmate is able to figure out the answer for themselves. It is unacceptable to share any artifacts, such as code and/or write-ups for this assignment. If you work with someone in close collaboration, you must mention your collaborator on your assignment.

\emph{Suggested practice problems, from CLRS:} 22.1-1 through 22.1-5; 22.1-6 (challenge); 22.2-3; 22.2-6; 22.-3-6; 22.3-8; 22.3-11; 22.4-2; 22.4-5; 22.5-3; 22.5-4

\begin{enumerate}

\item \textbf{(From problem set 5; previous exam question)} Let $A[1..n]$ be an array of non-integers taken from some set $K$ of size $k>1$. \emph{(Note: For this problem, you are not given the set $K$ or $k$; this is only to illustrate that there are $k$ distinct non-integer numbers. We only have access to elements through $A$. Further, note that $k$ may be small or large: from constant to even larger than $n$.)}
\begin{enumerate}
\item Describe an algorithm that sorts $A$ in expected time $O(n + k\lg k)$, and describe why it has this running time. 

\item What is the worst-case running time of your algorithm? Justify your answer.
\end{enumerate}

\item Let $G=(V,E)$ be an $n$-vertex undirected graph consisting of \emph{cops} and \emph{robbers} as vertices. You are given two facts about the graph:
\begin{enumerate}
\item $G$ is connected, and
\item each edge is incident to exactly one cop and one robber. (That is, no edge is incident to two cops, and no edge is incident to two robbers.)
\end{enumerate}
Suppose we know that Dave $\in V$ is a cop. Give an efficient algorithm to distinguish all cops from all robbers.


\item Problem 22.2-5 from CLRS.

\item Problem 22.5-1 from CLRS. In addition to stating your answer, also (formally) prove its correctness.

\item Problem 22.5-7 from CLRS.


\end{enumerate}

\end{document}
