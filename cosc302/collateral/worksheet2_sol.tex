\documentclass[letterpaper,11pt]{article}

\usepackage[margin=1in]{geometry}
\usepackage{amsmath}
\usepackage{amsthm}
\usepackage{amsfonts}

\newif\ifSolutions
\Solutionsfalse
%\Solutionstrue

\usepackage{fancyhdr}
\pagestyle{fancy}
\lhead{COSC 302 --- {\bf Worksheet 2: Asymptotic Analysis and Proofs (solutions)} --- Spring 2018}
\rhead{}


\newtheorem{theorem}{Theorem}[section]

\begin{document}
\thispagestyle{plain}
\noindent{COSC 302: Analysis of Algorithms --- Spring 2018}

\noindent{Prof. Darren Strash}

\noindent{Colgate University} \\

\noindent{\bf Worksheet 2 --- Asymptotic Analysis and Proofs (solutions)} \\

\begin{enumerate}
\item Prove that $n^2 + n + 5 = \Theta(n^2)$.\\

We show that $n^2 + n + 5 = O(n^2)$ and that $n^2 + n + 5 = \Omega(n^2)$. By Theorem 3.1, this shows that $n^2 + n + 5 = \Theta(n^2)$.

First we show that $n^2 + n + 5 = O(n^2)$.
\begin{proof}
We show that $\exists c>0,n_0>0 \text{ s.t. } 0 \leq n^2 + n + 5 \leq cn^2$, $\forall n \geq n_0$.
\begin{align*}
n^2 + n + 5 &\leq n^2 + n + 5n^2&n\geq 1\\
            &\leq n^2 + n^2 + 5n^2&n\geq 1\\
            &=    7n^2&n\geq 1
\end{align*}
Furthermore, we have that $n^2 + n + 5 \geq 0$ for $n\geq 1$.
Therefore for $c=7$, and all $n\geq n_0=1$, we have that $0\leq n^2 + n + 5 \leq cn^2$ and by definition $n^2 + n + 5 = O(n^2)$.
\end{proof}

Next, we show that $n^2 + n + 5 = \Omega(n^2)$.
\begin{proof}
We show that $\exists c>0,n_0>0 \text{ s.t. } 0 \leq cn^2 \leq n^2 + n + 5$, $\forall n \geq n_0$.
\begin{align*}
n^2 + n + 5 &\geq n^2 + n & \forall n\\
            &\geq n^2 + n^2&n\geq 1\\
            &=    2n^2&n\geq 1\\
            &\geq 0 &n\geq 1
\end{align*}
Furthermore, we have that $n^2 + n + 5 \geq 0$ for $n\geq 1$.
Therefore for $c=2$, and all $n\geq n_0=1$, we have that $0\leq cn^2 \leq n^2 + n + 5$ and by definition $n^2 + n + 5 = \Omega(n^2)$.
\end{proof}
\item \emph{(Reflexivity)} Prove that if $f(n)$ is asymptotically non-negative, then $f(n) = O(f(n))$.\\
\textbf{Solution:}
\begin{proof}
First note that since $f(n)$ is asymptotically non-negative, $\exists n_1 > 0$ such that $f(n) \geq 0$ $\forall n\geq n_1$. Then,
\begin{align*}
0 &\leq f(n)& n \geq n_1\\
  &\leq 1\cdot f(n) & n \geq n_1\\
\end{align*}
Therefore for $c=1$ and $n_0=n_1$, $0\leq f(n) \leq cf(n)$ for all $n\geq n_0$ and $f(n) = O(f(n))$.
\end{proof}

\newpage
\item Prove that $\lg(n!) = \Theta(n \lg n)$.\\
\textbf{Solution:}
\begin{proof}
We show that $\lg(n!) = O(n\lg n)$ and $\lg(n!)= \Omega(n\lg n)$. By Theorem 3.1, this shows that $\lg(n!)=\Theta(n\lg n)$.

($\lg(n!) = O(n\lg n)$):
\begin{align*}
\lg(n!) &= \lg(n\cdot (n-1)\cdots 2\cdot 1)&\\
        &\leq\lg(n\cdot n \cdots n\cdot n) & \text{for $n \geq 1$}\\
        &=\lg(n^n) &\\
        &=n\lg n. &
\end{align*}
Therefore for $c=1$, and all $n\geq n_0=1$, we have that $0\leq \lg(n!) \leq cn\lg n$, and $\lg(n!) = O(n\lg n)$.\\

($\lg(n!) = \Omega(n\lg n)$):
\begin{align*}
\lg(n!) &= \lg(n\cdot (n-1)\cdots (n/2+1)\cdots2\cdot 1)&\\
        &\geq \lg(n\cdot (n-1)\cdots (n/2+1))& \text{when $n\geq 4$}\\
        &\geq \lg(\frac{n}{2}\cdot \frac{n}{2} \cdots \frac{n}{2}) & \text{$n/2$ times}\\
        &=\frac{n}{2}\lg(\frac{n}{2}) &\\
        &=\frac{n}{2}\lg{n} - \frac{n}{2} &\\
        &\geq\frac{n}{2}\lg{n} - \frac{n}{2}\lg{\sqrt{n}} &\text{$1 \leq \lg{\sqrt{n}}$, when $n\geq 4$}&\\
        &=\frac{n}{2}\lg{n} - \frac{n}{4}\lg{n} &\\
        &=\frac{n}{4}\lg{n}&
\end{align*}
Therefore for $c=1/4$, and all $n\geq n_0=4$, we have that $0\leq c\lg(n!) \leq n\lg n$, and $\lg(n!) = \Omega(n\lg n)$.
\end{proof}
\item \emph{(Transitivity)} Let $f(n)$, $g(n)$, and $h(n)$ be asymptotically non-negative functions. Prove that if $f(n) = O(g(n))$ and $g(n) = O(h(n))$, then $f(n) = O(h(n))$.\\
\textbf{Solution:}
\begin{proof}
Let $f(n) = O(g(n))$, then there are positive constants $c_1,n_1$ such that $0\leq f(n) \leq c_1g(n)$ for all $n\geq n_1$.
Further, let $g(n) = O(h(n))$, then there are positive constants $c_2,n_2$ such that $0\leq g(n) \leq c_2h(n)$ for all $n\geq n_2$.

Then,
\begin{align*}
0 &\leq f(n)&n \geq n_1 \text{\quad(since $f(n) = O(g(n))$)} \\
  &\leq c_1g(n)&n \geq n_1 \text{\quad(since $f(n) = O(g(n))$)} \\
  &\leq c_1c_2h(n)& n \geq \max\{n_1,n_2\} \text{\quad(since $g(n) = O(h(n))$)}
\end{align*}
Therefore, for $c=c_1c_2$ and $n_0 = \max\{n_1,n_2\}$ we have that $0 \leq f(n) \leq ch(n)$ and $f(n) = O(h(n))$.
\end{proof}

\newpage
\item \emph{(Reflexivity)} Prove that if $f(n)$ is asymptotically non-negative, then $f(n) = O(f(n))$.\\
\emph{(Edit: This is a repeat of question 2. Instead we will show that $f(n) = \Omega(f(n))$.}

\begin{proof}
First note that since $f(n)$ is asymptotically non-negative, $\exists n_1 > 0$ such that $f(n) \geq 0$ $\forall n\geq n_1$. Then,
\begin{align*}
0 &\leq 1\cdot f(n)& n \geq n_1\\
  &\leq f(n) & n \geq n_1\\
\end{align*}

Therefore for $c=1$ and $n_0=n_1$, $0\leq cf(n) \leq f(n)$ for all $n\geq n_0$ and $f(n) = \Omega(f(n))$.
\end{proof}

\item \emph{(Transpose symmetry)} Let $f(n)$ and $g(n)$ be two asymptotically non-negative functions. Prove that if $f(n) = O(g(n))$ then $g(n) = \Omega(f(n))$.\\
\textbf{Solution:}
\begin{proof}
We first show the forward direction, that is: if $f(n) = O(g(n))$ then $g(n) = \Omega(f(n))$.

($\Rightarrow$) Suppose $f(n) = O(g(n))$, then there are positive constants $c_1,n_1$ such that $0\leq f(n) \leq c_1g(n)$ for all $n\geq n_1$.

Since $c_1>0$, it is also the case that $0\leq \frac{1}{c_1}f(n) \leq g(n)$ for all $n\geq n_1$. Therefore, for constants $c=\frac{1}{c_1}>0, n_0 = n_1>0$, we have that $0\leq cf(n) \leq g(n)$. Hence, $g(n) = \Omega(f(n))$.\\

Now we show the reverse direction, that is: if $g(n) = \Omega(f(n))$, then $f(n) = O(g(n))$. The proof is similar.

($\Leftarrow$) Suppose $g(n) = \Omega(f(n))$, then there are positive constants $c_1,n_1$ such that $0\leq c_1f(n) \leq g(n)$ for all $n\geq n_1$.

Since $c_1>0$, it is also the case that $0\leq f(n) \leq \frac{1}{c_1}g(n)$ for all $n\geq n_1$. Therefore, for constants $c=\frac{1}{c_1}>0, n_0 = n_1>0$, we have that $0\leq f(n) \leq cg(n)$. Hence, $f(n) = O(g(n))$.
\end{proof}

\newpage
\item Prove that $n^3 \neq O(n^2)$ by contradiction.\\

\textbf{Solution:}
\begin{proof}
Suppose, for sake of contradiction that $n^3 = O(n^2)$. That is, there are constants $c>0,n_0>0$ such that $0\leq n^3\leq cn^2$ for all $n\geq n_0$.

Let $x = \max\{n_0,c+1\}$. For $n \geq x$, $xn^2 \leq n^3$, and by definition of $O(\cdot)$, $n^3 \leq cn^2$. This implies that $xn^2 \leq cn^2$, but $xn^2 > cn^2$---a contradiction.
\end{proof}

\vspace*{6cm}
\item Prove that, for $n \in \mathbb{N},\,\,\sum_{i=0}^{n}i = \frac{n(n+1)}{2}$ by induction.\\
\emph{(Edit: The original version of this question said ``for $i\in \mathbb{N}$'')}\\

\textbf{Solution:}
\begin{proof}
Base case: $n=0$, then both $\sum_{i=0}^{0}i = 0$ and $\frac{0(0+1)}{2} = 0$.\\

Inductive step: Assume that $\sum_{i=0}^{k-1}i = \frac{(k-1)k}{2}$. Then
\begin{align*}
\sum_{i=0}^{k}i &= \sum_{i=0}^{k-1}i + k&\\
                &= \frac{(k-1)k}{2} + k&\text{I.H.}\\
                &= \frac{k^2-k + 2k}{2}&\\
                &= \frac{k(k+1)}{2}.&
\end{align*}
\end{proof}

\newpage
\item Prove that, for $c\neq 1$ and $n \in \mathbb{N},\,\,\sum_{i=0}^{n}c^i = \frac{c^{n+1} - 1}{c - 1}$ by induction.\\

\textbf{Solution:}
\begin{proof}
Base case: For $n=0$, then 
\[\sum_{i=0}^{0}c^i = c^0 = 1 = \frac{c-1}{c-1} = \frac{c^1 - 1}{c - 1}.\]

Inductive step: Assume that $\sum_{i=0}^{k-1}c^i = \frac{c^{k} - 1}{c - 1}$. Then,
\begin{align*}
\sum_{i=0}^{k}i &= \sum_{i=0}^{k}c^i\\
                &= \sum_{i=0}^{k-1}c^i + c^k\\
                &= \frac{c^{k} - 1}{c - 1} + c^k & \text{I.H.}\\
                &= \frac{c^{k} - 1}{c - 1} + c^k\frac{c-1}{c-1} \\
                &= \frac{c^{k} - 1}{c - 1} + \frac{c^{k+1}-c^{k}}{c-1} \\
                &= \frac{c^{k+1} + c^k - c^k - 1}{c - 1} \\
                &= \frac{c^{k+1} - 1}{c - 1}.
\end{align*}
\end{proof}
\newpage
\item Prove the following by induction: Given an unlimited supply of 5-cent stamps and 7-cent stamps, we can make any amount of postage larger than 23 cents.\\

\textbf{Solution:}
\begin{proof}
Let $p$ be the amount of postage we can make. We have two cases: $23 < p \leq 28$, and $p > 28$.

Case 1: We begin with $p > 28$.
We inductively assume that we can make $p-5$ cents worth of postage (which is greater than $23$). Then we add a $5$-cent stamp to get $p = (p-5) + 5$ cents of postage.

Case 2: We now show it is true for $23 < p \leq 28$, by manually showing that each value $p$ is a sum of $5$'s and $7$'s. 

$p = 24 = 7 + 7 + 5 + 5$

$p = 25 = 5 + 5 + 5 + 5 + 5$

$p = 26 = 7+7+7+5$

$p = 27 = 7 + 5 + 5 + 5 + 5$

$p = 28 = 7 + 7 + 7 + 7$
\end{proof}
\end{enumerate}
\end{document}
