\documentclass[letterpaper,11pt]{article}

\newif\ifSolutions
%\Solutionsfalse
\Solutionstrue


\usepackage[margin=1.0in]{geometry}

%\usepackage{amsmath}
\usepackage{mathtools}
\usepackage{amsthm}
\usepackage{amsfonts}
\usepackage{url}
\usepackage{mathdots}

%\newtheorem{theorem}{Theorem}[section]
\newtheorem*{invariant*}{Invariant}

\usepackage{fancyhdr}
\pagestyle{fancy}
\lhead{{\bf Worksheet 7 --- Graphs; Traversals \& Applications; Minimum Spanning Trees (with Solutions)}}
\rhead{}

\usepackage[noend]{algpseudocode}
\usepackage{algorithm}

\algrenewcommand\algorithmicdo{}
\algrenewcommand{\algorithmiccomment}[1]{// #1}

\newcommand{\midd}{\ensuremath{\mathrm{mid}}}
\newcommand{\E}{\ensuremath{\mathrm{E}}}
\newcommand{\Prob}{\ensuremath{\mathrm{Pr}}}
\newcommand{\argmin}{\ensuremath{\mathrm{arg\,min}}}
\newcommand{\argmax}{\ensuremath{\mathrm{arg\,max}}}

\begin{document}
\thispagestyle{plain}
\noindent{COSC 302: Analysis of Algorithms Lecture --- Spring 2018}

\noindent{Prof. Darren Strash}

\noindent{Colgate University} \\

\noindent{\bf Worksheet 7 --- Graphs; Traversals \& Applications; Minimum Spanning Trees} \\

\section{Graphs}
\begin{enumerate}
\item Give the formal definition of a graph.\\

\textbf{Solution:} A graph $G=(V,E)$ is an ordered pair consistning of a set of vertices $V$ and a set of edges $E\subseteq V\times V$.

\item A graph can be formed between any objects that have pairwise relationships. Describe how to form graphs from objects that you encounter every day. What do the vertices and edges represent?\\

\textbf{Solution:} 
\begin{itemize}
\item Friendship networks; there is a vertex for each person, and an edge exists between two vertices iff their respective people are friends.
\item ``Who follows whom'' in the Twitter network; again a vertex for each person, and an edge $(u,x)$ is in the graph iff $u$ follows $x$.
\item Road networks; there is a vertex for each intersection, and edges for each road segment connecting intersections.
\end{itemize}

\item What is the maximum number of edges that an undirected graph on $n$ vertices can have?\\

\textbf{Solution:} This is the number of ways to choose $2$ items from $n$ items. That is, $\binom{n}{2} = \frac{n(n-1)}{2}$.

\item What is the running time of breadth-first search if you are given the adjacency matrix representation (and not the adjacency list) of the graph?\\

\textbf{Solution:} When removing a vertex from the queue, we evaluate all of its neighbors. In an adjacency matrix, this takes $O(n)$ time. This is done $n$ times, taking overall $O(n^2)$ time.

\item What is a tree and how does it differ from a rooted tree and an ordered tree?\\

\textbf{Solution:} A graph $G = (V,E)$ is a tree iff
\begin{itemize}
\item $G$ is connected,
\item $G$ has $|V| - 1$ edges, and
\item $G$ has no cycles.
\end{itemize}

Noe that, if any two of these are true, this implies the third is true.

A rooted tree has a distinguished node called the \emph{root}, which induces a parent/child relationship between vertices. An ordered tree is rooted, and children have a predefined order. A heap is an example of a rooted tree, a binary search tree is an example of an ordered tree.

\end{enumerate}

\section{Graph Traversals and Applications}

\begin{enumerate}
\setcounter{enumi}{5}
\item Suppose you are in a maze and you wish to find the closest exit. Describe an efficient algorithm to solve this problem.\\

\textbf{Solution}: Model the maze as a graph. If you have access to a map of the maze, then run BFS on the graph, ending when encountering the first exit. This is the closet exit.

\item Describe an algorithm to decide if a given graph is a tree. What is the running time of your algorithm?\\

\textbf{Solution:} Verify two of the three properties listed above for trees:
\begin{itemize}
\item Check that $G$ is connected. Run BFS and check that it reaches all vertices in the graph.
\item Check to see if the number of edges is $|V|-1$. This can by done by iterating over the adjacency list and counting the edges in $\Theta(n + m)$ time.
\item Check that $G$ has no cycles. Run DFS on the input graph. If DFS yields any non-tree edges then the graph has a cycle.
\end{itemize}
\newpage
\item Describe an algorithm to compute an Euler tour in $O(n + m)$ time.\\

\textbf{Solution:} Run DFS. Every recursive call introduces a new forward edge, returning from the recursive call introduces a return edge. When encountering a non-tree edge, introduce two edges, one out and back.

%\item Describe how to adjust the graph to have the same strongly connected components while only having $O(n)$ edges.
\item Prove by contradiction that the component graph is a dag.\\

\textbf{Solution:} 
\begin{proof}
Assume for sake of contradiction that $G$ is a component graph and not a dag. We will show that $G$ is not a component graph.

Since $G$ is a component graph, it is directed, and each vertex $V_i$ corresponds to a strongly connected component, which is a set of vertices in some other graph $G' = (V',E')$.

If $G$ is not a dag, then it contains a cycle $V_1,V_2,V_3,\ldots,V_k,V_1$. Then $\exists (v_1,v_2)\in E'$, such that $v_1\in V_1, v_2\in V_2$.

Note then that there is a path from $v_1$ to $v_2$. Note that there is also a path from $v_2$ to $v_1$, that passes through vertex sets $V_2$,$V_3$,$\ldots$,$V_k$,$V_1$. Thus, $V_1\cup V_2\cup\cdots V_k$ is strongly connected, and $V_1$,$V_2$ are not strongly connected components. Therefore, $G$ is not a component graph.
\end{proof}
\end{enumerate}

\section{Minimum Spanning Trees}
\begin{enumerate}
\setcounter{enumi}{9}
\item Give a graph that does not have a unique minimum spanning tree.\\

\textbf{Solution:} A cycle where every edge has weight one.

\item Describe an algorithm to determine if a graph has a unique minimum spanning tree.\\

\textbf{Solution:} First compute a minimum spanning tree (MST) $T = (V_T,E_T)$. For each edge in $E_T$, remove the edge, compute a minimum spanning tree of the remaining graph, and if it has the same weight as $T$, then the MST is not unique; otherwise return $e$ to the graph and continue. If no MST (computed in this way) has the same weight as $T$ then the MST is unique.

\item Describe how to implement the Prim-Jarn\'ik algorithm to run in $O((n + m)\lg n)$ time.

\emph{(To be discussed in lab.)}
\end{enumerate}
\end{document}

